\documentclass[a4paper]{article}
\usepackage[utf8]{inputenc}
\usepackage[czech]{babel}
\usepackage{setspace}
\usepackage{hyperref}
\usepackage[usenames,dvipsnames]{color}

\definecolor{src-bck}{RGB}{240, 240, 240}

\DeclareUrlCommand\url{\def\UrlLeft{<}\def\UrlRight{>} \urlstyle{tt}}

\hypersetup{
    bookmarks=true,         % show bookmarks bar?
    unicode=true,           % non-Latin characters in Acrobat bookmarks
    pdftoolbar=true,        % show Acrobat toolbar?
    pdfmenubar=true,        % show Acrobat menu?
    pdffitwindow=false,     % window fit to page when opened
    pdfstartview={FitH},    % fits the width of the page to the window
    pdftitle={Bakalářská práce},    
                            % title
    pdfauthor={Jan Papoušek},% author
    pdfsubject={Paralelizace metod pro analýzu dynamických systémů pomocí grafické karty},
                            % subject of the document    
    pdfkeywords={keyword1} {key2} {key3}, % list of keywords
    pdfnewwindow=true,      % links in new window
    colorlinks=false,       % false: boxed links; true: colored links
    linkcolor=blue,         % color of internal links
    citecolor=green,        % color of links to bibliography
    filecolor=magenta,      % color of file links
    urlcolor=cyan,          % color of external links
    linkbordercolor={1 1 0}
}

\begin{document}
\doublespacing

\section{Anotace}
Projekt si klade za cíl vytvoření nástroje pro analýzu dynamických systémů modelovaných pomocí obyčejných diferenciálních rovnic. Na rozdíl od existujících
nástrojů pro simulaci chování dynamických systémů a monitoring temporálních vlastností nad běhy těchto systémů je zde podstatný aspekt vzniku modulární
architektury, jež by umožňovala budoucí vývoj a testování optimalizace a paralelizace jednotlivých modulů i analytických algoritmů.

\section{Cíle}
\begin{enumerate}
\item   V~současné době je projekt složen z~většího množství samostatných modulů,
        implementujících některé funkce výsledného nástroje. Tyto moduly vznikly
        na základě \cite{drazan_master}, kde byla prezentována základní myšlenka
        uvažované metody pro analýzu dynamických systémů, která je rozšířena o~přístup
        prezentován v~\cite{odkaz na popis robustnosti}. Největší pozornost je přitom
        věnována simulaci chování a monitoringu. Tyto části se jeví být výpočetně
        nejvíce náročné a možný způsob jejich optimalizace je již nastíněn v~\cite{papousek_bachelor}
        a \cite{kovacik_bachelor}. Jistý pokrok již byl učiněn i v~zadávání temporálních
        vlastností~\cite{vejpustek_bachelor} použitých v~rámci analýzy.

        Aktuální stav je již nyní možné sledovat na \url{https://github.com/sybila/parasim}.

\item   Projekt si klade za úkol vytvořit nástroj pro analýzu dynamických systémů
        pocházejích zejména z oblasti biologie.

        Již existující samostatné moduly je nutné integrovat a rozšířit o další
        nutnou funkcionalitu (načítání modelu a vlastností, vizualizace napočítaných
        výsledků a exportování výsledků do souboru). Do budoucna se počítá s~dalšími
        algoritmy pro analýzu dynamických systémů, a proto návrh a implementace
        architektury umožní integrovat další rozšíření. To otevře prostor pro účast
        třetích stran na vývoji.

        V první verzi bude mít nástroj pouze rozhraní z příkazové řádky. Pro zvýšení
        přístupnosti však musí nástroj obsah i grafické uživatelské rozhraní. Není
        v~plánu vytvořit editor modelů (takové nástroje jsou již k~dispozici),
        avšak je pravděpodobné, že bude obsah editor temporálních vlastností~\cite{vejpustek_bachelor}. 
        Klíčovou součástí grafického rozhraní jsou:
        \begin{itemize}
            \item   zobrazení výsledků analýzy,
            \item   sdružení a správa modelů, vlastností a výsledků.
        \end{itemize}

        Po úspěšné integraci jednotlivých modulů bude nutné se zaměřit na akceleraci
        výpočtu. Na základě naměřených dat se rozhodne, zda se bude akcelerovat
        nějaký z~modulů samostatně např. za použití technologie CUDA~\cite{cuda_doc},
        nebo zda se výpočet bude spouštět v distribuovaném prostředí např. pomocí frameworku
        Hadoop~\cite{hadoop_book}.

        Aby bylo podpořeno rozšíření mezi potenciální uživatele, bude nástroj
        umístěn na \cite{parasim_web} a doplněn o~uživatelskou dokumantací a tutorialem.
        Program bude licencován pod \emph{[GNU GPL/Apache License]}, jejíž volnost
        by měla podpořit rozšiřování nástroje třetí stranou.

%\item Projekt si klade za úkol vytvořit fungující nástroj určený pro analýzu dynamických systémů, který který bude cílen především na uživatele-neinformatiky.
%
%	Jednotlivé samostatné moduly je nutné integrovat a doplnit o~další funkcionalitu \emph{[moduly]} nutnou pro základní činnost nástroje (například načítání
%	vstupů). Protože se v~budoucnu počítá s~rozšířeními \emph{[zopakovat to, co Sven na začátku?]}, jeví se vhodné navrhnout a implementovat architekturu, která umožní
%	jednoduchou a standardizovanou integraci další funkcionality. To zároveň umožní účast třetí strany na jejím vývoji.
%
%	Aby mohl být nástroj používán i neinformatiky, je jeho nutnou součásti grafické uživatelské rozhraní (GUI). Projekt si neklade za cíl implementovat editor modelů
%	nebo temporálních 	vlasností \emph{[sic]}, což by neodpovídalo jeho zaměření na analýzu. Navíc již řada grafických editorů existuje. Klíčové je však přehledné
%	nastavení analýzy a zobrazení jejích výsledků. Zároveň by nástroj mohl sdružovat vstupy, se kterými uživatel pracuje, stejně jako výsledky jeho práce.
%	\emph{[mohl by -- proboha -- tak bude sdružovat nebo ne? celkově -- spíše \uv{bude umět}, nebo \uv{měl by umět}?]} Protože architektura nástroje je modulární,
%	mělo by i GUI počítat s možností zavedení dalších rozšíření. \emph{[tak a teď si nejsem jistý, jestli tohle fakt chci dělat a jestli to bude nutné; dále viz výše]}
%
%	\emph{[Ten konec možná postavit jinak -- podle Papiho vize -- že modulární GUI nebude třeba. Ale pak se musí zapsat i na začátek.]}
%
%	Výsledný nástroj bude schopný provést simulaci dynamického systému a analyzovat jej na základě monitoringu temporálních vlastností. Další funkcionalitu bude přitom
%	jednouché zavést na základě architektury jeho jádra. Grafické rozhraní bude uschovávat množinu možných vstupů analýzy (modelů, temporálních vlastností, počátečních podmínek, etc.),
%	které bude moci uživatel vybrat a spustit na jejich základě analýzu, jejíž výsledek nástroj zobrazí a uchová pro další referenci.
%
%	Aby bylo podpořeno rozšíření mezi potenciální uživatele, bude nástroj umístěn na \emph{[kde -- jen github a nebo sourceforge?]} a doplněn \emph{[alespoň základní?]} uživatelskou %%dokumentací
%	a tutorialem. Program bude licencován pod \emph{[GNU GPL/Apache License]}, jejíž volnost by měla podpořit rozšiřování nástroje třetí stranou. \emph{[tady to je stylově trochu %zvláštní]}

%	\emph{GUI pro extenze? Nebo je to spíše programátorská záležitost?}

\item\begin{enumerate}
	\item duben, květen\\
		Základní náplní bude integrace a dopisování modulů. Výstupem by měl být prototyp nástroje spustitelný z~příkazového řádku.
        \begin{itemize}
            \item   Jan Papoušek -- integrace modulů, vizualizace výsledků
            \item   Tomáš Vejpustek -- načítání vstupů, reprezentace dat, exportování výsledků
        \end{itemize}
	\item červen, červenec\\
		Na základě prototypu proběhne testování a profilování existujících modulů.
		Zároveň bude vytvořena dokumentace prototypu ve formě manuálové stránky s~příklady použití.
        \begin{itemize}
            \item   Jan Papoušek -- profilování, dokumentace
            \item   Tomáš Vejpustek -- testování, dokumentace
        \end{itemize}
	\item srpen, září, říjen\\ \emph{[tady je otázka, jestli to napsat každému zvlášť nebo to spojit]}
		Bude věnována návrhu a implementaci GUI. To by mělo na konci etapy vykazovat většinu navrhované funkcionality.
        \begin{itemize}
            \item   Jan Papoušek -- optimalizace výpočtu
            \item   Tomáš Vejpustek -- návrh a implementace GUI
        \end{itemize}
	\item listopad, prosinec\\
		Proběhne testování a dokončování grafického rozhraní. Zároveň bude vytvořena jeho uživatelská dokumentace a tutorial.
        \begin{itemize}
            \item   Jan Papoušek -- testy, další profilování, dokumentace
            \item   Tomáš Vejpustek -- testy, dokončení GUI, dokumentace
        \end{itemize}
	\end{enumerate}
\end{enumerate}

\bibliographystyle{plain}
\bibliography{project}

\end{document}
