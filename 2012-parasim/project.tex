\documentclass[a4paper]{article}
\usepackage[utf8]{inputenc}
\usepackage[czech]{babel}
\usepackage{setspace}
\usepackage{hyperref}
\usepackage[usenames,dvipsnames]{color}

\definecolor{src-bck}{RGB}{240, 240, 240}

\DeclareUrlCommand\url{\def\UrlLeft{<}\def\UrlRight{>} \urlstyle{tt}}

\hypersetup{
    bookmarks=true,         % show bookmarks bar?
    unicode=true,           % non-Latin characters in Acrobat bookmarks
    pdftoolbar=true,        % show Acrobat toolbar?
    pdfmenubar=true,        % show Acrobat menu?
    pdffitwindow=false,     % window fit to page when opened
    pdfstartview={FitH},    % fits the width of the page to the window
    pdftitle={Bakalářská práce},    
                            % title
    pdfauthor={Jan Papoušek},% author
    pdfsubject={Paralelizace metod pro analýzu dynamických systémů pomocí grafické karty},
                            % subject of the document    
    pdfkeywords={keyword1} {key2} {key3}, % list of keywords
    pdfnewwindow=true,      % links in new window
    colorlinks=false,       % false: boxed links; true: colored links
    linkcolor=blue,         % color of internal links
    citecolor=green,        % color of links to bibliography
    filecolor=magenta,      % color of file links
    urlcolor=cyan,          % color of external links
    linkbordercolor={1 1 0}
}

\begin{document}
\doublespacing

\section{Anotace}
Projekt si klade za cíl vytvoření nástroje pro analýzu dynamických systémů modelovaných pomocí obyčejných diferenciálních rovnic. Na rozdíl od existujících
nástrojů pro simulaci chování dynamických systémů a monitoring temporálních vlastností nad běhy těchto systémů je zde podstatný aspekt vzniku modulární
architektury, jež by umožňovala budoucí vývoj a testování optimalizace a paralelizace jednotlivých modulů i analytických algoritmů.

\section{Cíle}
\begin{enumerate}

\item   Řešená úloha je ověření, za jakých iniciálních podmínek splňuje
        dynamický systém danou temporálních vlastnost. Výchozím bodem je algoritmus
        prezentovaný v \cite{drazan_master} a dále rozšířen na základě \cite{robustness}.

        Problém byl rozdělen na podúkoly a jim příslušné moduly:
        \begin{enumerate}
            \item   simulace systému diferenciálních rovnic,
            \item   detekce cyklu,
            \item   monitoring temporální vlastnosti,
            \item   určování vzdálenosti mezi trajektoriemi chování,
            \item   zahušťování prostoru iniciálních podmínek.
        \end{enumerate}

        Největší pozornost byla do této doby věnována modulům (a) a (c). Ty se jeví
        být výpočetně nejnáročnější a možný způsob jejich optimalizace byl nastíněn
        v \cite{papousek_bachelor} a \cite{kovacik_bachelor}. Je možné také navázat
        na vizuální zadávání temporálních vlastností prezentované v \cite{vejpustek_bachelor}.
        
        Aktuální stav je již nyní možné sledovat na \url{https://github.com/sybila/parasim}.

\item   Projekt si klade za úkol vytvořit nástroj pro analýzu dynamických systémů
        pocházejících zejména z oblasti biologie.

        Již existující samostatné moduly je nutné integrovat a rozšířit o další
        funkcionalitu (načítání modelu a vlastností, vizualizace napočítaných
        výsledků a exportování výsledků do souboru). Do budoucna se počítá s~dalšími
        algoritmy pro analýzu dynamických systémů, a proto návrh a implementace
        architektury umožní integrovat další rozšíření. To otevře prostor pro účast
        třetích stran na vývoji.

        V první verzi bude mít nástroj pouze rozhraní z příkazového řádku. Pro zvýšení
        přístupnosti však musí nástroj nabídnout i grafické uživatelské rozhraní. Není
        v~plánu vytvořit editor modelů (takové nástroje jsou již k~dispozici, např. Copasi, CellDesigner),
        avšak je pravděpodobné, že bude obsahovat editor temporálních vlastností. 
        Klíčovými součástmi grafického rozhraní jsou:
        \begin{itemize}
            \item   zobrazení výsledků analýzy,
            \item   sdružení a správa modelů, vlastností a výsledků.
        \end{itemize}

        Po úspěšné integraci jednotlivých modulů bude nutné se zaměřit na akceleraci
        výpočtu. Na základě naměřených dat se rozhodne, zda se bude akcelerovat
        nějaký z~modulů samostatně např. za použití technologie CUDA,
        nebo zda se výpočet bude spouštět v distribuovaném prostředí např. pomocí frameworku
        Hadoop.

        Aby bylo podpořeno rozšíření mezi potenciální uživatele, bude nástroj
        umístěn na \url{https://github.com/sybila/parasim} a doplněn o~uživatelskou
        dokumantaci a tutoriál.
        Program bude licencován pod GNU GPL, jejíž volnost
        by měla podpořit rozšiřování nástroje třetí stranou.

\item\begin{enumerate}
	\item duben, květen\\
		Základní pracovní náplní bude integrace a dopisování modulů. Výstupem bude
        prototyp nástroje spustitelný z~příkazového řádku.
        \begin{itemize}
            \item   Jan Papoušek -- integrace modulů, vizualizace výsledků
            \item   Tomáš Vejpustek -- načítání vstupů, reprezentace dat, exportování výsledků
        \end{itemize}
	\item červen, červenec\\
		Na základě prototypu proběhne testování a profilování existujících modulů.
		Zároveň bude vytvořena dokumentace prototypu ve formě manuálové stránky s~příklady použití.
        \begin{itemize}
            \item   Jan Papoušek -- profilování, dokumentace
            \item   Tomáš Vejpustek -- testy, dokumentace
        \end{itemize}
	\item srpen, září, říjen\\
		Třetí etapa bude věnována návrhu a implementaci GUI společně s optimalizací časově
        náročných častí výpočtu. Optimalizace bude přitom vycházet z výsledků profilování.
        Na konci etapy bude nástroj vykazovat většinu navrhované funkcionality.
        \begin{itemize}
            \item   Jan Papoušek -- návrh optimalizace výpočtu a její implementace
            \item   Tomáš Vejpustek -- návrh a implementace GUI
        \end{itemize}
	\item listopad, prosinec\\
		Proběhne vyhodnocení optimalizace, testování a dokončování nástroje.
        Zároveň bude vytvořena uživatelská dokumentace.
        \begin{itemize}
            \item   Jan Papoušek -- testy, další profilování, dokumentace
            \item   Tomáš Vejpustek -- testy, dokončení GUI, dokumentace
        \end{itemize}
	\end{enumerate}
\end{enumerate}

\bibliographystyle{plain}
\bibliography{project}

\end{document}
