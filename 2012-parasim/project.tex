\documentclass[a4paper]{article}
\usepackage[utf8]{inputenc}
\usepackage[czech]{babel}
\usepackage{setspace}

\begin{document}
\doublespacing

\section{Anotace (Sven)}
Projekt si klade za cíl vytvoření nástroje pro analýzu dynamických systémů modelovaných pomocí obyčejných diferenciálních rovnic. Na rozdíl od existujících
nástrojů pro simulaci chování dynamických systémů a monitoring temporálních vlastností nad běhy těchto systémů je zde podstatný aspekt vzniku modulární
architektury, jež by umožňovala budoucí vývoj a testování optimalizace a paralelizace jednotlivých modulů i analytických algoritmů.

\section{Cíle (T.)}
\textit{Není lepší nástroj nějak pojmenovat? slovo \uv{nástroj} už působí nešikovně.}
\begin{enumerate}
\item V~současné době je projekt složen z~většího množství samostatných modulů, implementujících některé funkci výsledného nástroje. Největší pozornost
	je přitom věnována simulaci chování a monitoringu.

	\begin{itemize}\it
	\item rozvést to, že nespolupracují?
	\item psát, jestli se to váže na bakalářku a tak?
	\end{itemize}

\item Projekt si klade za úkol vytvořit fungující nástroj určený pro analýzu dynamických systémů, který který bude cílen především na uživatele-neinformatiky.

	Jednotlivé samostatné moduly je nutné integrovat a doplnit o~další funkcionalitu \emph{[moduly]} nutnou pro základní činnost nástroje (například načítání
	vstupů). Protože se v~budoucnu počítá s~rozšířeními \emph{[zopakovat to, co Sven na začátku?]}, jeví se vhodné navrhnout a implementovat architekturu, která umožní
	jednoduchou a standardizovanou integraci další funkcionality. To zároveň umožní účast třetí strany na jejím vývoji.

	Aby mohl být nástroj používán i neinformatiky, je jeho nutnou součásti grafické uživatelské rozhraní (GUI). Projekt si neklade za cíl implementovat editor modelů
	nebo temporálních 	vlasností \emph{[sic]}, což by neodpovídalo jeho zaměření na analýzu. Navíc již řada grafických editorů existuje. Klíčové je však přehledné
	nastavení analýzy a zobrazení jejích výsledků. Zároveň by nástroj mohl sdružovat vstupy, se kterými uživatel pracuje, stejně jako výsledky jeho práce.
	\emph{[mohl by -- proboha -- tak bude sdružovat nebo ne? celkově -- spíše \uv{bude umět}, nebo \uv{měl by umět}?]} Protože architektura nástroje je modulární,
	mělo by i GUI počítat s možností zavedení dalších rozšíření. \emph{[tak a teď si nejsem jistý, jestli tohle fakt chci dělat a jestli to bude nutné; dále viz výše]}

	\emph{[Ten konec možná postavit jinak -- podle Papiho vize -- že modulární GUI nebude třeba. Ale pak se musí zapsat i na začátek.]}

	Výsledný nástroj bude schopný provést simulaci dynamického systému a analyzovat jej na základě monitoringu temporálních vlastností. Další funkcionalitu bude přitom
	jednouché zavést na základě architektury jeho jádra. Grafické rozhraní bude uschovávat množinu možných vstupů analýzy (modelů, temporálních vlastností, počátečních podmínek, etc.),
	které bude moci uživatel vybrat a spustit na jejich základě analýzu, jejíž výsledek nástroj zobrazí a uchová pro další referenci.

	Aby bylo podpořeno rozšíření mezi potenciální uživatele, bude nástroj umístěn na \emph{[kde -- jen github a nebo sourceforge?]} a doplněn \emph{[alespoň základní?]} uživatelskou dokumentací
	a tutorialem. Program bude licencován pod \emph{[GNU GPL/Apache License]}, jejíž volnost by měla podpořit rozšiřování nástroje třetí stranou. \emph{[tady to je stylově trochu zvláštní]}

	\emph{GUI pro extenze? Nebo je to spíše programátorská záležitost?}

\item\begin{enumerate}
	\item duben, květen\\
		Základní náplní bude integrace a dopisování modulů. Výstupem by měl být prototyp nástroje spustitelný z~příkazového řádku \emph{[vykazující základní funkcionalitu nástroje,
		tedy numerickou simulaci a monitoring]}.
	\item červen, červenec\\
		Na základě prototypu proběhne testování a profilování existujících modulů.
		Zároveň bude vytvořena dokumentace prototypu ve formě manuálové stránky s~příklady použití.
	\item srpen, září, říjen\\ \emph{[tady je otázka, jestli to napsat každému zvlášť nebo to spojit]}
		Bude věnována návrhu a implementaci GUI. To by mělo na konci etapy vykazovat většinu navrhované funkcionality.
	\item listopad, prosinec\\
		Proběhne testování a dokončování grafického rozhraní. Zároveň bude vytvořena jeho uživatelská dokumentace a tutorial.
	\end{enumerate}
\end{enumerate}

\end{document}
